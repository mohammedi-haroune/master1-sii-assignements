\documentclass[•]{article}

\usepackage{../../TP0/style}

\begin{document}
\def\reportnumber{5}
\def\reporttitle{TRIGGERS}
%----------------------------------------------------------------------------------------
%	TITLE PAGE
%----------------------------------------------------------------------------------------

\begin{titlepage} % Suppresses displaying the page number on the title page and the subsequent page counts as page 1
	\newcommand{\HRule}{\rule{\linewidth}{0.5mm}} % Defines a new command for horizontal lines, change thickness here
	
	\center % Centre everything on the page
	
	%------------------------------------------------
	%	Headings
	%------------------------------------------------
	
	\baselineskip=2\baselineskip 
	\textsc{\LARGE Université des Sciences et de la Technologie Houari Boumediene}%\\[1cm] % Main heading such as the name of your university/college

	%------------------------------------------------
	%	Logo
	%------------------------------------------------
	
	%\vfill\vfill
	\vfill
	\includegraphics[width=0.3\textwidth]{USTHB_Logo.png}\\[1cm] % Include a department/university logo - this will require the graphicx package
	 
	%----------------------------------------------------------------------------------------
	
	\textsc{\Large Compilation}\\[0.5cm] % Major heading such as course name
	%\textsc{\large Minor Heading}\\[0.5cm] % Minor heading such as course title
	
	%------------------------------------------------
	%	Title
	%------------------------------------------------
	
	\HRule\\[0.4cm]
	\baselineskip=1.2\baselineskip 
	{\huge\bfseries Travaux Pratique  \textdegree  \reportnumber \\ \reporttitle}\\[0.4cm] % Title of your document
	
	\HRule\\[1.5cm]
	
	%------------------------------------------------
	%	Author(s)
	%------------------------------------------------
	
		\begin{center}
			\large
			\textit{Professeur}\\
			% votre non ICI
		\end{center}
	
	%------------------------------------------------
	%	Date
	%------------------------------------------------
	
	\vfill\vfill\vfill % Position the date 3/4 down the remaining page
	
	{\large\today} % Date, change the \today to a set date if you want to be precise
	
	
	\vfill % Push the date up 1/4 of the remaining page
	
\end{titlepage}



Avant de commencer à répondre aux questions demandées, il faut exécuter la commande \texttt{SERVEROUTPUT ON}, elle permet d'activer les affichages résultants de la fonction \texttt{dbms\_output.put\_line();} .

\subsection{Création d'un trigger qui affiche \texttt{"un nouveau client est ajouté"} après chaque insertion d’un client,
Répétez la même chose pour la modification ou la suppression.}
Dans ce cas la, nous avons a sélectionner plusieurs lignes (tuples), donc nous utiliserons un curseur afin de parcourir les éléments sélectionnés un par un et puis les traiter.
Aussi si la table \texttt{MARQUE} est vide, une exception doit être généré, affichant un message correspondant à l'erreur rencontrée.

\subsubsection{Requête}
\begin{sql}
CREATE OR REPLACE TRIGGER INSERT_CLIENT 
AFTER INSERT ON CLIENT 
FOR EACH ROW
BEGIN
	dbms_output.put_line('un nouveau client est ajouté');
END;
/


CREATE OR REPLACE TRIGGER UPDATE_CLIENT 
AFTER UPDATE ON CLIENT 
FOR EACH ROW
BEGIN
	dbms_output.put_line('Les informations d un client sont mises a jours');
END;
/


CREATE OR REPLACE TRIGGER DELETE_CLIENT 
AFTER DELETE ON CLIENT 
FOR EACH ROW
BEGIN
	dbms_output.put_line('un client est supprimé');
END;
/
\end{sql}

Ainsi chaque ligne affichée en sortie donne le nom de la marque ainsi que le nombre de modèles qu'elle possède.
\subsubsection{Résultat}
\begin{sql}
Trigger created.

Trigger created.

Trigger created.
\end{sql}

\subsection{Tester les trigger}
d'abord activer les affichages avec la commande \texttt{SET SERVEROUTPUT ON}

\subsubsection{Requête}
\begin{sql}
/** l'insertion **/
INSERT INTO CLIENT 
VALUES(40,'MME','Adjaout','feriel',to_date('21/01/1995', 'DD/MM/YYYY'),'cite 2110 logts bt 5 alger','0561381877','0565568716','');

/** la modification **/
UPDATE CLIENT 
SET PRENOMCLIENT = 'ADJAOUTE' 
WHERE NUMCLIENT = 40 ;

/** la suppression **/
DELETE FROM CLIENT 
WHERE NUMCLIENT = 40 ;

\end{sql}
\subsubsection{Résultat}
\begin{sql}
un nouveau client est ajouté
1 row created.

Les informations d un client sont mises a jours
1 row updated.

un client est supprimé
1 row deleted.
\end{sql}

\subsection{Création du trigger qui afficher \texttt{"un nouveau modèle est ajouté à la marque [Nom de la marque]"}
après chaque insertion d'un}

\begin{itemize}
\item
La clause \texttt{INTO} dans la requête \texttt{SELECT} nous permet de recupérer la 
valeur de l'attribut en question dans une variable
\item
Le tuple spécial \texttt{:NEW} générer par le SGBD dans le corps les triggers des insertions 
sauvgarde le tuple qui viens d'être inserer dans la base
\item
En utilisant les concepts \texttt{INTO} et \texttt{:NEW}, on écrit la requete qui récupère la marque du modèle en question
\end{itemize}


\subsubsection{Requête}
\begin{sql}
CREATE OR REPLACE TRIGGER NOUVEAU_MODELE
AFTER INSERT ON MODELE
FOR EACH ROW
DECLARE 
NOM_MARQUE MARQUE.MARQUE%TYPE;
NUM MARQUE.NUMMARQUE%TYPE;

BEGIN
    NUM := :NEW.NUMMARQUE ;

	SELECT MARQUE INTO NOM_MARQUE
	FROM MARQUE WHERE NUMMARQUE = NUM;


	DBMS_OUTPUT.PUT_LINE('un nouveau modele est ajoute a la marque' || NOM_MARQUE);
END;
/
\end{sql}


\subsubsection{Résultat}
\begin{sql}
Trigger created.
\end{sql}


\subsection{Tester les triggers}
\subsubsection{Requête}
\begin{sql}
INSERT INTO MODELE VALUES (99,15,'504');
\end{sql}


\subsubsection{Résultat}
\begin{sql}
un nouveau modele est ajoute a la marque peugeot
1 row inserted.
\end{sql}

%Question 3
\begin{itemize}
\item
Le \texttt{BEFORE} permet de remettre en cause la mise a jours contenu dans le trigger
\item
Pour cela l'on doit arreter la modification avant son exécution , donc utiliser le \texttt{BEFORE}
puis l'on compare la nouvelle valeur du \texttt{SALAIRE} que l'on souhaite mettre "\texttt{NEW}" avec l'ancienne existante "\texttt{OLD}"
et l'on génère une interruption (erreur) avec la fonction \texttt{raise\_application\_error} pour empecher la modification dans le cas ou
\texttt{NEW < OLD}
\end{itemize}

\subsubsection{Requête}
\begin{sql}
CREATE OR REPLACE TRIGGER UPDATE_SALAIRE
BEFORE UPDATE OF SALAIRE ON EMPLOYE 
FOR EACH ROW
BEGIN
	IF :NEW.SALAIRE < :OLD.SALAIRE
	THEN raise_application_error(-20001, 'Le nouveau salaire ne doit pas etre inférieure a l`ancien!') ;
	END IF;
END;
/ 
\end{sql}
\subsubsection{Résultat}
\begin{sql}
Trigger created.
\end{sql}

\subsection{Tester les triggers}
Pour pouvoir s'assurer que le trigger \texttt{UPDATE\_SALAIRE} sont fonctionelles on va insérer un tuple qui génère une exception
\subsubsection{Requête}
\begin{sql}
UPDATE EMPLOYE SET SALAIRE = SALAIRE - 100 WHERE NUMEMPLOYE = 53 ;
\end{sql}

\subsubsection{Résultat}
\begin{sql}
ERROR at line 1:
ORA-20001: Le nouveau salaire ne doit pas etre inférieure a l`ancien!
ORA-06512: at "SYSTEM.UPDATE_SALAIRE", line 3
ORA-04088: error during execution of trigger 'SYSTEM.UPDATE_SALAIRE'
\end{sql}

Le trigger a arreté le programme et invalidé la mise a jours, Il a ainsi protégé l'integrité de la base de données

\subsection{QUESTION 5}
Afin que l’administrateur puisse connaitre le nombre total des interventions pour chaque employé. 
Pour cela, nous ajoutons l'attribut TOTAL\_INTERVENTIONS dans la table employé.
Puis nous créons le trigger TOTAL\_INTERVENTIONS\_TRIGGER qui met à jour l’attribut TOTAL\_INTERVENTIONS.


\subsubsection{Requête}
\begin{sql}
ALTER TABLE EMPLOYE ADD (TOTAL_INTERVENTIONS INTEGER DEFAULT 0);
\end{sql}

\subsubsection{Résultat}
\begin{sql}
Table altered.
\end{sql}

le trigger suivant incrémente la valeur de l'attribut TOTAL\_INTERVENTIONS lorsqu'un employé participe a une nouvelle intervention,
c'est à dire lorsque un element (tuple) est inséré dans la table INTERVENANT

\subsubsection{Requête}
\begin{sql}
CREATE OR REPLACE TRIGGER TOTAL_INTERVENTIONS_TRIGGER
AFTER INSERT ON INTERVENANT
FOR EACH ROW
BEGIN
	UPDATE EMPLOYE E
	SET TOTAL_INTERVENTIONS = TOTAL_INTERVENTIONS + 1 
	WHERE E.NUMEMPLOYE = :NEW.NUMEMPLOYE ;
END;
/
\end{sql}

\subsubsection{Résultat}
\begin{sql}
Trigger created.
\end{sql}

\subsection{Tester les triggers}
Pour pouvoir s'assurer que le trigger \texttt{UPDATE\_SALAIRE} sont fonctionelles on va afficher le nomber d'intervention avant et après l'insertion d'une intervention d'un employé donné


\subsubsection{Requête}
\begin{sql}
SELECT TOTAL_INTERVENTIONS FROM EMPLOYE WHERE NUMEMPLOYE = 53;
INSERT INTO INTERVENANT VALUES (1,53,to_date('2017-01-27 08:00:00','YYYY-MM-DD HH24:MI:SS' ),to_date('2017-02-09 14:00:00','YYYY-MM-DD HH24:MI:SS' ));
SELECT TOTAL_INTERVENTIONS FROM EMPLOYE WHERE NUMEMPLOYE = 53;
\end{sql}

\subsubsection{Résultat}
\begin{sql}
TOTAL_INTERVENTIONS
-------------------
                  0

1 row created.

TOTAL_INTERVENTIONS
-------------------
                  1
\end{sql}

Effectivement après l'insertion le nombre totale d'intervention à été incrémenté
\end{document}
