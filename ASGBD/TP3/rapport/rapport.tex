%arara : pdflatex
\documentclass[•]{article}

\usepackage{../../TP0/style}

\begin{document}
\def\reportnumber{3}
\def\reporttitle{Dictionnaire Oracle}
%----------------------------------------------------------------------------------------
%	TITLE PAGE
%----------------------------------------------------------------------------------------

\begin{titlepage} % Suppresses displaying the page number on the title page and the subsequent page counts as page 1
	\newcommand{\HRule}{\rule{\linewidth}{0.5mm}} % Defines a new command for horizontal lines, change thickness here
	
	\center % Centre everything on the page
	
	%------------------------------------------------
	%	Headings
	%------------------------------------------------
	
	\baselineskip=2\baselineskip 
	\textsc{\LARGE Université des Sciences et de la Technologie Houari Boumediene}%\\[1cm] % Main heading such as the name of your university/college

	%------------------------------------------------
	%	Logo
	%------------------------------------------------
	
	%\vfill\vfill
	\vfill
	\includegraphics[width=0.3\textwidth]{USTHB_Logo.png}\\[1cm] % Include a department/university logo - this will require the graphicx package
	 
	%----------------------------------------------------------------------------------------
	
	\textsc{\Large Compilation}\\[0.5cm] % Major heading such as course name
	%\textsc{\large Minor Heading}\\[0.5cm] % Minor heading such as course title
	
	%------------------------------------------------
	%	Title
	%------------------------------------------------
	
	\HRule\\[0.4cm]
	\baselineskip=1.2\baselineskip 
	{\huge\bfseries Travaux Pratique  \textdegree  \reportnumber \\ \reporttitle}\\[0.4cm] % Title of your document
	
	\HRule\\[1.5cm]
	
	%------------------------------------------------
	%	Author(s)
	%------------------------------------------------
	
		\begin{center}
			\large
			\textit{Professeur}\\
			% votre non ICI
		\end{center}
	
	%------------------------------------------------
	%	Date
	%------------------------------------------------
	
	\vfill\vfill\vfill % Position the date 3/4 down the remaining page
	
	{\large\today} % Date, change the \today to a set date if you want to be precise
	
	
	\vfill % Push the date up 1/4 of the remaining page
	
\end{titlepage}


\subsection{Connecter en tant que « System ». Lister le catalogue « DICT ». Il contient combien d’instances. Donner sa
structure}

\subsubsection{Requête}
\begin{sql}
CONNECT SYSTEM/oracle
SELECT COUNT(*) AS NBR FROM DICT;
DESC DICT;

\end{sql}
\subsubsection{Résultat}
\begin{sql}
       NBR
----------
      2551

Name                                      Null?    Type
----------------------------------------- -------- ------------------
TABLE_NAME                                         VARCHAR2(30)
COMMENTS                                           VARCHAR2(4000)
\end{sql}


\subsection{Donner le rôle et la structure des tables (ou vues) suivantes : \texttt{ALL\_TAB\_COLUMNS},
\texttt{USER\_USERS}, \texttt{ALL\_CONSTRAINTS} et \texttt{USER\_TAB\_PRIVS}}
pour donner le rôle d'une table on conslulte le dictionaire \texttt{DICT}, plus précesement sa colone \texttt{COMMENTS}
\subsubsection{Requête}
\begin{sql}
SELECT * FROM DICT 
WHERE TABLE\_NAME IN ('ALL\_TAB\_COLUMNS','USER\_USERS', 'ALL\_CONSTRAINTS', 'USER\_TAB\_PRIVS');
\end{sql}
\subsubsection{Résultat}
\begin{sql}
TABLE_NAME		       COMMENTS
------------------------------ -------------------------------------------------
ALL_CONSTRAINTS 	       Constraint definitions on accessible tables
ALL_TAB_COLUMNS 	       Columns of user's tables, views and clusters
USER_TAB_PRIVS		       Grants on objects for which the user is the owner
USER_USERS		       	   Information about the current user
\end{sql}
\subsection{Trouver le nom d’utilisateur avec lequel vous êtes connecté}

Pour trouver le nom de l'utilisateur il faut consulter la table \texttt{USER\_USERS} et il faut tout d'abord afficher son schema avec la command \texttt{DESC} et après on cosulute la table pour nous donner le nom de l'utlisateur

\subsubsection{Requête}
\begin{sql}
DESC USER\_USERS;
SELECT USERNAME FROM USER\_USERS; 
\end{sql}
\subsubsection{Résultat}
\begin{sql}
 Name                                      Null?    Type
 ----------------------------------------- -------- -----------------

 USERNAME                                  NOT NULL VARCHAR2(30)
 USER_ID                                   NOT NULL NUMBER
 ACCOUNT_STATUS                            NOT NULL VARCHAR2(32)
 LOCK_DATE                                          DATE
 EXPIRY_DATE                                        DATE
 DEFAULT_TABLESPACE                        NOT NULL VARCHAR2(30)
 TEMPORARY_TABLESPACE                      NOT NULL VARCHAR2(30)
 CREATED                                   NOT NULL DATE
 INITIAL_RSRC_CONSUMER_GROUP                        VARCHAR2(30)
 EXTERNAL_NAME                                      VARCHAR2(4000)

 USERNAME
 ------------------------------
 SYSTEM
\end{sql}

%4
\subsection{Comparer la structure et le contenu des tables \texttt{ALL\_TAB\_COLUMNS} et \texttt{USER\_TAB\_COLUMNS}}
\subsubsection{Requête}
\begin{sql}
DESC ALL_TAB_COLUMNS;
DESC USER_TAB_COLUMNS;
SELECT COUNT(*) AS NBR_ALL_TAB_COLUMNS FROM ALL_TAB_COLUMNS;
SELECT COUNT(*) AS NBR_USER_TAB_COLUMNS FROM USER_TAB_COLUMNS;
\end{sql}
\subsubsection{Résultat}
\begin{sql}
 Name                                      Null?    Type
 ----------------------------------------- -------- ----------------------------
 OWNER                                     NOT NULL VARCHAR2(30)
 TABLE_NAME                                NOT NULL VARCHAR2(30)
 COLUMN_NAME                               NOT NULL VARCHAR2(30)
 DATA_TYPE                                          VARCHAR2(106)
 DATA_TYPE_MOD                                      VARCHAR2(3)
 DATA_TYPE_OWNER                                    VARCHAR2(120)
 DATA_LENGTH                               NOT NULL NUMBER
 DATA_PRECISION                                     NUMBER
 DATA_SCALE                                         NUMBER
 NULLABLE                                           VARCHAR2(1)
 COLUMN_ID                                          NUMBER
 DEFAULT_LENGTH                                     NUMBER
 DATA_DEFAULT                                       LONG
 NUM_DISTINCT                                       NUMBER
 LOW_VALUE                                          RAW(32)
 HIGH_VALUE                                         RAW(32)
 DENSITY                                            NUMBER
 NUM_NULLS                                          NUMBER
 NUM_BUCKETS                                        NUMBER
 LAST_ANALYZED                                      DATE
 SAMPLE_SIZE                                        NUMBER
 CHARACTER_SET_NAME                                 VARCHAR2(44)
 CHAR_COL_DECL_LENGTH                               NUMBER
 GLOBAL_STATS                                       VARCHAR2(3)
 USER_STATS                                         VARCHAR2(3)
 AVG_COL_LEN                                        NUMBER
 CHAR_LENGTH                                        NUMBER
 CHAR_USED                                          VARCHAR2(1)
 V80_FMT_IMAGE                                      VARCHAR2(3)
 DATA_UPGRADED                                      VARCHAR2(3)
 HISTOGRAM                                          VARCHAR2(15)

 Name                                      Null?    Type
 ----------------------------------------- -------- ----------------------------
 TABLE_NAME                                NOT NULL VARCHAR2(30)
 COLUMN_NAME                               NOT NULL VARCHAR2(30)
 DATA_TYPE                                          VARCHAR2(106)
 DATA_TYPE_MOD                                      VARCHAR2(3)
 DATA_TYPE_OWNER                                    VARCHAR2(120)
 DATA_LENGTH                               NOT NULL NUMBER
 DATA_PRECISION                                     NUMBER
 DATA_SCALE                                         NUMBER
 NULLABLE                                           VARCHAR2(1)
 COLUMN_ID                                          NUMBER
 DEFAULT_LENGTH                                     NUMBER
 DATA_DEFAULT                                       LONG
 NUM_DISTINCT                                       NUMBER
 LOW_VALUE                                          RAW(32)
 HIGH_VALUE                                         RAW(32)
 DENSITY                                            NUMBER
 NUM_NULLS                                          NUMBER
 NUM_BUCKETS                                        NUMBER
 LAST_ANALYZED                                      DATE
 SAMPLE_SIZE                                        NUMBER
 CHARACTER_SET_NAME                                 VARCHAR2(44)
 CHAR_COL_DECL_LENGTH                               NUMBER
 GLOBAL_STATS                                       VARCHAR2(3)
 USER_STATS                                         VARCHAR2(3)
 AVG_COL_LEN                                        NUMBER
 CHAR_LENGTH                                        NUMBER
 CHAR_USED                                          VARCHAR2(1)
 V80_FMT_IMAGE                                      VARCHAR2(3)
 DATA_UPGRADED                                      VARCHAR2(3)
 HISTOGRAM                                          VARCHAR2(15)



NBR_ALL_TAB_COLUMNS                                                             
-------------------                                                             
              73142

NBR_USER_TAB_COLUMNS                                                            
--------------------                                                            
                1684
\end{sql}
La structure des tables est presque le même, la suele difference c'est l'attribut \texttt{OWNER} dans la table \texttt{ALL\_TAB\_COLUMNS}

La table \texttt{ALL\_TAB\_COLUMNS} contient beaucoup de lignes par rapport à la table \texttt{USER\_TAB\_COLUMNS}
ceci est justifier par le fait que cette denière contient que les information concernant les tables de l'utilisateur actualle 
tandis que la première contient  les information de toutes tables que l'utlisateur a un acces

%5
\subsection{Vérifiez que les tables du TP1 ont été réellement créées. Donner toutes les informations sur ces tables}
\subsubsection{Requête}
\begin{sql}
SELECT TABLE_NAME  FROM ALL_TABLES WHERE OWNER = 'DBAINTERVENTION' ;
\end{sql}
\subsubsection{Résultat}
\begin{sql}
TABLE_NAME
------------------------------
CLIENT
EMPLOYE
MARQUE
MODELE
VEHICULE
INTERVENTIONS
INTERVENANT

7 rows selected.
\end{sql}
\subsubsection{Requête}

\begin{sql}
SELECT * FROM ALL_TABLES WHERE OWNER = 'DBAINTERVENTION';
\end{sql}
Vu que le nombre d'infos est très important nous allons afficher que 3 colonnes qui sont: le nom de la table , le nom de la table space dans la quelle les tables ont été crées et le nom du proporiétaire de la table
\begin{sql}
SELECT TABLE_NAME, TABLESPACE_NAME, OWNER FROM ALL_TABLES WHERE OWNER = 'DBAINTERVENTION';
\end{sql}

\subsubsection{Résultat}
\begin{sql}
TABLE_NAME                     TABLESPACE_NAME				OWNER 
-------------------------------------------------------------------

CLIENT                         INTERVENTION_TBS				DBAINTERVENTION

EMPLOYE                        INTERVENTION_TBS				DBAINTERVENTION

MARQUE                         INTERVENTION_TBS				DBAINTERVENTION

MODELE                         INTERVENTION_TBS				DBAINTERVENTION

VEHICULE                       INTERVENTION_TBS				DBAINTERVENTION

INTERVENTIONS                  INTERVENTION_TBS				DBAINTERVENTION

INTERVENANT                    INTERVENTION_TBS				DBAINTERVENTION


7 rows selected.
\end{sql}

%6
\subsection{Lister les tables de l’utilisateur \texttt{SYSTEM} et celles de l’utilisateur \texttt{DBAINTERVENTION}}
\subsubsection{Requête}
\begin{sql}
SELECT TABLE_NAME  FROM ALL_TABLES WHERE OWNER = 'SYSTEM' ;
SELECT TABLE_NAME  FROM ALL_TABLES WHERE OWNER = 'DBAINTERVENTION' ;\end{sql}
\subsubsection{Résultat}
\begin{sql}
PAS DE RESULTATS
\end{sql}
L'attribut \texttt{OWNER} dans la table \texttt{ALL\_TABLES} contient le nom de la créateur de la table concerné

%7
\subsection{Donner la description des attributs des tables \texttt{VEHICULE} et \texttt{INTERVENTIONS}}
Vu que la quantité d'infos est grande nous nous contenterons d'afficher uniquement 5 infos qui sont :
\subsubsection{Requête}
\begin{sql}
DESC USER_TAB_COLUMNS;
SELECT * FROM USER_TAB_COLUMNS WHERE (TABLE_NAME = 'VEHICULE' OR TABLE_NAME = 'INTERVENTIONS');
\end{sql}
\subsubsection{Résultat}
\begin{sql}
 Name                                      Null?    Type
 ----------------------------------------- -------- -----------------------

 TABLE_NAME                                NOT NULL VARCHAR2(30)
 COLUMN_NAME                               NOT NULL VARCHAR2(30)
 DATA_TYPE                                          VARCHAR2(106)
 DATA_TYPE_MOD                                      VARCHAR2(3)
 DATA_TYPE_OWNER                                    VARCHAR2(120)
 DATA_LENGTH                               NOT NULL NUMBER
 DATA_PRECISION                                     NUMBER
 DATA_SCALE                                         NUMBER
 NULLABLE                                           VARCHAR2(1)
 COLUMN_ID                                          NUMBER
 DEFAULT_LENGTH                                     NUMBER
 DATA_DEFAULT                                       LONG
 NUM_DISTINCT                                       NUMBER
 LOW_VALUE                                          RAW(32)
 HIGH_VALUE                                         RAW(32)
 DENSITY                                            NUMBER
 NUM_NULLS                                          NUMBER
 NUM_BUCKETS                                        NUMBER
 LAST_ANALYZED                                      DATE
 SAMPLE_SIZE                                        NUMBER
 CHARACTER_SET_NAME                                 VARCHAR2(44)
 CHAR_COL_DECL_LENGTH                               NUMBER
 GLOBAL_STATS                                       VARCHAR2(3)
 USER_STATS                                         VARCHAR2(3)
 AVG_COL_LEN                                        NUMBER
 CHAR_LENGTH                                        NUMBER
 CHAR_USED                                          VARCHAR2(1)
 V80_FMT_IMAGE                                      VARCHAR2(3)
 DATA_UPGRADED                                      VARCHAR2(3)
 HISTOGRAM                                          VARCHAR2(15)

 TABLE_NAME                     COLUMN_NAME                    N		DATA_TYPE		DATA_LENGTH
---------------------------------------------------------------------------------------------------

INTERVENTIONS                   NUMINTERVENTION                N		NUMBER	         22

INTERVENTIONS                   NUMVEHICULE                    Y		NUMBER	         22

INTERVENTIONS                   TYPEINTERVENTION               Y		VARCHAR2         40

INTERVENTIONS                   DATEDEBINTERV                  Y		DATE	          7

INTERVENTIONS                   DATEFININTERV                  Y		DATE	          7

INTERVENTIONS                   COUTINTERV                     Y		FLOAT	         22

VEHICULE                        NUMVEHICULE                    N		NUMBER	         22

VEHICULE                        NUMCLIENT                      Y		NUMBER	         22

VEHICULE                        NUMMODELE                      Y		NUMBER	         22

VEHICULE                        NUMIMMAT                       Y		NUMBER	         22

VEHICULE                        ANNEE                          Y		VARCHAR2	      4

11 rows selected.
\end{sql}

%8
\subsection{Comment peut-on vérifie qu’il y a une référence de clé étrangère entre les tables \texttt{VEHICULE} et
\texttt{INTERVENTIONS}}
\begin{enumerate}
	\item Chercher dans le dictionaire touts les tables qui contient le mot \texttt{CONSTRAINTS}
	\begin{sql}
SELECT * FROM DICT WHERE TABLE_NAME LIKE '%CONSTRAINTS%';
TABLE_NAME		       COMMENTS
------------------------------ -------------------------------------------------
DBA_CONSTRAINTS 	       Constraint definitions on all tables
ALL_CONSTRAINTS 	       Constraint definitions on accessible tables
USER_CONSTRAINTS	       Constraint definitions on user's own tables
	\end{sql}
	\item A partir de la description des tables on extrait les noms des tables qui peuvent contenir l'information concerntant la containte demandé, dans ce cas : \texttt{USER\_CONSTRAINTS} puis afficher son scheams
	\begin{sql}
DESC USER_CONSTRAINTS;
 Name                                      Null?    Type
 ----------------------------------------- -------- ----------------------------
 OWNER                                              VARCHAR2(120)
 CONSTRAINT_NAME                           NOT NULL VARCHAR2(30)
 CONSTRAINT_TYPE                                    VARCHAR2(1)
 TABLE_NAME                                NOT NULL VARCHAR2(30)
 SEARCH_CONDITION                                   LONG
 R_OWNER                                            VARCHAR2(120)
 R_CONSTRAINT_NAME                                  VARCHAR2(30)
 DELETE_RULE                                        VARCHAR2(9)
 STATUS                                             VARCHAR2(8)
 DEFERRABLE                                         VARCHAR2(14)
 DEFERRED                                           VARCHAR2(9)
 VALIDATED                                          VARCHAR2(13)
 GENERATED                                          VARCHAR2(14)
 BAD                                                VARCHAR2(3)
 RELY                                               VARCHAR2(4)
 LAST_CHANGE                                        DATE
 INDEX_OWNER                                        VARCHAR2(30)
 INDEX_NAME                                         VARCHAR2(30)
 INVALID                                            VARCHAR2(7)
 VIEW_RELATED                                       VARCHAR2(14)
	\end{sql}
	\item Ecriture de la reqûte qui vérifie l'existense du contrainte.
	\begin{sql}
 SELECT *
 FROM USER_CONSTRAINTS
 WHERE CONSTRAINT_TYPE = 'R' AND TABLE_NAME = 'VEHICULE' OR TABLE_NAME = 'INTERVENTIONS';
PAS DE RESULTATS
	\end{sql}
\end{enumerate}

%9
\subsection{Donner toutes les contraintes créées lors du TP1 et les informations qui les caractérisent}
En raison du nbr imporant d'info nous n'afficherons que les colonnes suivants: \texttt{CONSTRAINT\_NAME}, \texttt{CONSTRAINT\_TYPE}, \texttt{TABLE\_NAME} et \texttt{STATUS}
\subsubsection{Requête}
\begin{sql}
DESC USER_CONSTRAINTS;
SELECT  CONSTRAINT_NAME, CONSTRAINT_TYPE, TABLE_NAME, STATUS  FROM USER_CONSTRAINTS WHERE TABLE_NAME IN ('EMPLOYE','VEHICULE','MARQUE','MODELE','INTERVENTIONS','INTERVENANT','CLIENT');
\end{sql}
\subsubsection{Résultat}
\begin{sql}
  Name                                      Null?    Type
 ----------------------------------------- -------- --------------------

 OWNER                                              VARCHAR2(120)
 CONSTRAINT_NAME                           NOT NULL VARCHAR2(30)
 CONSTRAINT_TYPE                                    VARCHAR2(1)
 TABLE_NAME                                NOT NULL VARCHAR2(30)
 SEARCH_CONDITION                                   LONG
 R_OWNER                                            VARCHAR2(120)
 R_CONSTRAINT_NAME                                  VARCHAR2(30)
 DELETE_RULE                                        VARCHAR2(9)
 STATUS                                             VARCHAR2(8)
 DEFERRABLE                                         VARCHAR2(14)
 DEFERRED                                           VARCHAR2(9)
 VALIDATED                                          VARCHAR2(13)
 GENERATED                                          VARCHAR2(14)
 BAD                                                VARCHAR2(3)
 RELY                                               VARCHAR2(4)
 LAST_CHANGE                                        DATE
 INDEX_OWNER                                        VARCHAR2(30)
 INDEX_NAME                                         VARCHAR2(30)
 INVALID                                            VARCHAR2(7)
 VIEW_RELATED                                       VARCHAR2(14)

CONSTRAINT_NAME                C TABLE_NAME                     STATUS
------------------------------ - ------------------------------ --------
CH_EMPLOYE                     C EMPLOYE                        ENABLED
FK_VEHICULE_MODELE             R VEHICULE                       ENABLED
FK_VEHICULE_CLIENT             R VEHICULE                       ENABLED
FK_MODELE_MARQUE               R MODELE                         ENABLED
FK_INTERVENTIONS_VEHICULE      R INTERVENTIONS                  ENABLED
FK_INTERVENANT_INTERVENTIONS   R INTERVENANT                    ENABLED
FK_INTERVENANT_EMPLOYE         R INTERVENANT                    ENABLED
PK_VEHICULE                    P VEHICULE                       ENABLED
PK_MODELE                      P MODELE                         ENABLED
PK_MARQUE                      P MARQUE                         ENABLED
PK_INTERVENTIONS               P INTERVENTIONS                  ENABLED
PK_INTERVENANT                 P INTERVENANT                    ENABLED
PK_EMPLOYE                     P EMPLOYE                        ENABLED
PK_CLIENT                      P CLIENT                         ENABLED
CHK_DATEINTERV                 C INTERVENTIONS                  DISABLED

15 rows selected.
\end{sql}

%10
\subsection{Retrouver toutes les informations permettant de recréer la table \texttt{INTERVENTIONS}}
Pour récrer la table \texttt{INTERVENTIONS} il faut avoir les informations suivantes : les colones et leurs types, toutes les contraintes associés, toutes indexes créere sur la table et enfin toutes les priveleges données pour les utilisateurs qui concerne cette table

\subsubsection{Requête}
\begin{sql}
 DESC INTERVENTIONS;
 SELECT * FROM ALL_TABLES WHERE TABLE_NAME = 'INTERVENTIONS';
 SELECT * FROM ALL_TAB_COLUMNS WHERE TABLE_NAME = 'INTERVENTIONS';
 SELECT * FROM ALL_CONSTRAINTS WHERE TABLE_NAME = 'INTERVENTIONS';
 SELECT * FROM ALL_INDEXES WHERE TABLE_NAME = 'INTERVENTIONS';
 SELECT * FROM ALL_IND_COLUMNS WHERE TABLE_NAME = 'INTERVENTIONS';
 SELECT * FROM ALL_TAB_PRIVS WHERE TABLE_NAME = 'INTERVENTIONS' ;
\end{sql}

\subsubsection{Résultat}
\begin{sql}
PAS DE RESULTATS
\end{sql}

%11
\subsection{Trouver tous les privilèges accordés à Admin.}
\subsubsection{Requête}
\begin{sql}
DESC USER_TAB_PRIVS;
SELECT PRIVILEGE, TABLE_NAME FROM USER_TAB_PRIVS WHERE GRANTEE = 'Admin';
\end{sql}
\subsubsection{Résultat}
\begin{sql}
  Name                                      Null?    Type
 ----------------------------------------- -------- --------------------

 GRANTEE                                   NOT NULL VARCHAR2(30)
 OWNER                                     NOT NULL VARCHAR2(30)
 TABLE_NAME                                NOT NULL VARCHAR2(30)
 GRANTOR                                   NOT NULL VARCHAR2(30)
 PRIVILEGE                                 NOT NULL VARCHAR2(40)
 GRANTABLE                                          VARCHAR2(3)
 HIERARCHY                                          VARCHAR2(3)

no rows selected
 \end{sql}

Il ya pas de privileges pour l'utilisateur \texttt{Admin} car nous avions retiré tous les privileges a l'Admin lors de TP précedent.

%12
\subsection{Trouver les rôles donnés à l’utilisateur Admin.}
\subsubsection{Requête}
\begin{sql}
DESC USER_ROLE_PRIVS;
CONNECT Admin/Admin;
SELECT GRANTED_ROLE FROM USER_ROLE_PRIVS;
\end{sql}
\subsubsection{Résultat}
\begin{sql}
 Name                                      Null?    Type
 ----------------------------------------- -------- ---------------------

 USERNAME                                           VARCHAR2(30)
 GRANTED_ROLE                                       VARCHAR2(30)
 ADMIN_OPTION                                       VARCHAR2(3)
 DEFAULT_ROLE                                       VARCHAR2(3)
 OS_GRANTED                                         VARCHAR2(3)

GRANTED_ROLE
------------------------------
GESTIONNAIRE_DES_INTERVENTIONS
 \end{sql}

%13
\subsection{Trouver tous les privilèges accordés à Admin.}
\subsubsection{Requête}
\begin{sql}
DESC USER_OBJECTS;
CONNECT ADMIN/ADMIN;
SELECT OBJECT_NAME FROM USER_OBJECTS ;
\end{sql}
\subsubsection{Résultat}
\begin{sql}
  Name                                      Null?    Type
 ----------------------------------------- -------- -------------------

 OBJECT_NAME                                        VARCHAR2(128)
 SUBOBJECT_NAME                                     VARCHAR2(30)
 OBJECT_ID                                          NUMBER
 DATA_OBJECT_ID                                     NUMBER
 OBJECT_TYPE                                        VARCHAR2(19)
 CREATED                                            DATE
 LAST_DDL_TIME                                      DATE
 TIMESTAMP                                          VARCHAR2(19)
 STATUS                                             VARCHAR2(7)
 TEMPORARY                                          VARCHAR2(1)
 GENERATED                                          VARCHAR2(1)
 SECONDARY                                          VARCHAR2(1)
 NAMESPACE                                          NUMBER
 EDITION_NAME                                       VARCHAR2(30)

OBJECT_NAME
---------------------------------------------

NOMEMP_IX
SYS_C007170
MY_TABLE
\end{sql}

%14
\subsection{L’administrateur cherche le propriétaire de la table INTERVENTIONS, comment il pourra le trouver}
\subsubsection{Requête}
\begin{sql}
SELECT OWNER FROM ALL_TABLES WHERE TABLE_NAME = 'INTERVENTIONS';
\end{sql}
\subsubsection{Résultat}
\begin{sql}
PAS DE RESULTATS
\end{sql}

%15
\subsection{Donner la taille en Ko de la table \texttt{INTERVENTIONS}}
\subsubsection{Requête}
\begin{sql}
SELECT BYTES/1024 AS Taille_Ko FROM USER_EXTENTS WHERE SEGMENT_NAME = 'INTERVENTIONS';
\end{sql}
\subsubsection{Résultat}
\begin{sql}
 TAILLE_KO
----------
        64
\end{sql}

%16
\subsection{Vérifier l’effet produit par chacune des commandes de définition de données du TP1 sur le dictionnaire.}
\subsubsection{Requête}
\begin{sql}
\end{sql}
\subsubsection{Résultat}
\begin{sql}
PAS DE RESULTATS
\end{sql}
\end{document}