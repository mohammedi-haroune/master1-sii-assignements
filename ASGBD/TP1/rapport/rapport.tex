%arara : pdflatex
\documentclass[•]{article}

\usepackage{../../TP0/style}

\begin{document}
\def\reportnumber{1}
\def\reporttitle{Création et manipulation d'une BD}
%----------------------------------------------------------------------------------------
%	TITLE PAGE
%----------------------------------------------------------------------------------------

\begin{titlepage} % Suppresses displaying the page number on the title page and the subsequent page counts as page 1
	\newcommand{\HRule}{\rule{\linewidth}{0.5mm}} % Defines a new command for horizontal lines, change thickness here
	
	\center % Centre everything on the page
	
	%------------------------------------------------
	%	Headings
	%------------------------------------------------
	
	\baselineskip=2\baselineskip 
	\textsc{\LARGE Université des Sciences et de la Technologie Houari Boumediene}%\\[1cm] % Main heading such as the name of your university/college

	%------------------------------------------------
	%	Logo
	%------------------------------------------------
	
	%\vfill\vfill
	\vfill
	\includegraphics[width=0.3\textwidth]{USTHB_Logo.png}\\[1cm] % Include a department/university logo - this will require the graphicx package
	 
	%----------------------------------------------------------------------------------------
	
	\textsc{\Large Compilation}\\[0.5cm] % Major heading such as course name
	%\textsc{\large Minor Heading}\\[0.5cm] % Minor heading such as course title
	
	%------------------------------------------------
	%	Title
	%------------------------------------------------
	
	\HRule\\[0.4cm]
	\baselineskip=1.2\baselineskip 
	{\huge\bfseries Travaux Pratique  \textdegree  \reportnumber \\ \reporttitle}\\[0.4cm] % Title of your document
	
	\HRule\\[1.5cm]
	
	%------------------------------------------------
	%	Author(s)
	%------------------------------------------------
	
		\begin{center}
			\large
			\textit{Professeur}\\
			% votre non ICI
		\end{center}
	
	%------------------------------------------------
	%	Date
	%------------------------------------------------
	
	\vfill\vfill\vfill % Position the date 3/4 down the remaining page
	
	{\large\today} % Date, change the \today to a set date if you want to be precise
	
	
	\vfill % Push the date up 1/4 of the remaining page
	
\end{titlepage}

\section{Création des tablespaces et des utilusateurs}
\subsection{Créer deux TableSpaces \texttt{INTERVENTION\_TBS} et \texttt{INTERVENTION\_TempTBS}}
\subsubsection{Requête}
\begin{sql}
CREATE TABLESPACE INTERVENTION_TBS
DATAFILE 'INTERVENTION_TBS.DAT'
SIZE 100M
AUTOEXTEND ON
ONLINE;

CREATE TEMPORARY TABLESPACE INTERVENTION_tempTBS
TEMPFILE 'INTERVENTION_tempTBS.DAT'
SIZE 100M
AUTOEXTEND ON;
\end{sql}
\subsubsection{Résultat}
\begin{sql}
Tablespace created.
Tablespace created.
\end{sql}
\subsection{Créer un utilisateur \texttt{DBAINTERVENTION} en lui attribuant les deux tablespaces créés précédemment}
\subsubsection{Requête}
\begin{sql}
CREATE USER DBAINTERVENTION
IDENTIFIED BY DBAINTERVENTION
DEFAULT TABLESPACE INTERVENTION_TBS
TEMPORARY TABLESPACE INTERVENTION_tempTBS;
\end{sql}
\subsubsection{Résultat}
\begin{sql}
User created.
\end{sql}
\subsection{Donner tous les privilèges à cet utilisateur.}
\subsubsection{Requête}
\begin{sql}
GRANT ALL PRIVILEGES TO DBAINTERVENTION;
\end{sql}
\subsubsection{Résultat}
\begin{sql}
Grant succeeded.
\end{sql}
\section{Langage de définition de données}
\subsection{Créer les relations de base avec toutes les contraintes d’intégrité}
\subsubsection{Requête}
\begin{sql}
CREATE TABLE CLIENT
(
NUMCLIENT INTEGER,
CIV VARCHAR2(40),
PRENOMCLIENT VARCHAR2(40),
NOMCLIENT VARCHAR2(40),
DATENAISSANCE DATE,
ADRESSE VARCHAR2(100),
TELPROF VARCHAR2(40),
TELPRIV VARCHAR2(40),
FAX VARCHAR2(40),
CONSTRAINT PK_CLIENT PRIMARY KEY (NUMCLIENT)
);

CREATE TABLE EMPLOYE
(
NUMEMPLOYE INTEGER,
PRENOMEMP VARCHAR2(40),
NOMEMP VARCHAR2(40),
CATEGORIE VARCHAR2(40),
SALAIRE REAL,
CONSTRAINT PK_EMPLOYE PRIMARY KEY (NUMEMPLOYE),
CONSTRAINT CH_EMPLOYE CHECK (CATEGORIE IN ('MECANICIEN' , 'ASSISTANT'))
);

CREATE TABLE MARQUE (
NUMMARQUE INTEGER,
MARQUE VARCHAR2(40),
PAYS VARCHAR2(40),
CONSTRAINT PK_MARQUE PRIMARY KEY (NUMMARQUE)
);

CREATE TABLE MODELE (
NUMMODELE INTEGER,
NUMMARQUE INTEGER,
MODELE VARCHAR2(40),
CONSTRAINT PK_MODELE PRIMARY KEY (NUMMODELE),
CONSTRAINT FK_MODELE_MARQUE FOREIGN KEY (NUMMARQUE) REFERENCES MARQUE(NUMMARQUE)
);

CREATE TABLE VEHICULE (
NUMVEHICULE INTEGER,
NUMCLIENT INTEGER,
NUMMODELE INTEGER,
NUMIMMAT INTEGER,
ANNEE VARCHAR2(4),
CONSTRAINT PK_VEHICULE PRIMARY KEY (NUMVEHICULE),
CONSTRAINT FK_VEHICULE_CLIENT FOREIGN KEY (NUMCLIENT) REFERENCES CLIENT(NUMCLIENT),
CONSTRAINT FK_VEHICULE_MODELE FOREIGN KEY (NUMMODELE) REFERENCES MODELE(NUMMODELE)
);

CREATE TABLE INTERVENTIONS (
NUMINTERVENTION INTEGER,
NUMVEHICULE INTEGER,
TYPEINTERVENTION VARCHAR2(40),
DATEDEBINTERV DATE,
DATEFININTERV DATE,
COUTINTERV REAL,
CONSTRAINT PK_INTERVENTIONS PRIMARY KEY (NUMINTERVENTION),
CONSTRAINT FK_INTERVENTIONS_VEHICULE FOREIGN KEY (NUMVEHICULE) REFERENCES VEHICULE(NUMVEHICULE)
);

CREATE TABLE INTERVENANT
(
NUMINTERVENTION INTEGER,
NUMEMPLOYE INTEGER,
DATEDEBUT DATE,
DATEFIN DATE,
CONSTRAINT PK_INTERVENANT PRIMARY KEY (NUMINTERVENTION, NUMEMPLOYE),
CONSTRAINT FK_INTERVENANT_EMPLOYE FOREIGN KEY (NUMEMPLOYE)REFERENCES EMPLOYE,
CONSTRAINT FK_INTERVENANT_INTERVENTIONS FOREIGN KEY (NUMINTERVENTION) REFERENCES INTERVENTIONS
);
\end{sql}
\subsubsection{Résultat}
\begin{sql}
Table created.
Table created.
Table created.
Table created.
Table created.
Table created.
Table created.
\end{sql}
\subsection{Ajouter l’attribut \texttt{DATEINSTALLATION} de type Date dans la relation \texttt{EMPLOYE}}
\subsubsection{Requête}
\begin{sql}
ALTER TABLE EMPLOYE ADD DATEINSTALLATION DATE NULL;
\end{sql}
\subsubsection{Résultat}
\begin{sql}
Table altered.
\end{sql}
\subsection{Ajouter la contrainte not null pour les attributs \texttt{CATEGORIE}, \texttt{SALAIRE} de la relation \texttt{EMPLOYE}}
\subsubsection{Requête}

\subsubsection{Résultat}
\begin{sql}
Table altered.
\end{sql}
\subsection{Modifier la longueur de l’attribut \texttt{PRENOMEMP} (agrandir, réduire)}
\subsubsection{Requête}
\begin{sql}
ALTER TABLE EMPLOYE MODIFY PRENOMEMP VARCHAR2(20);
ALTER TABLE EMPLOYE MODIFY PRENOMEMP VARCHAR2(50);
\end{sql}
\subsubsection{Résultat}
\begin{sql}
Table altered.
Table altered.
\end{sql}
\subsection{Supprimer la colonne \texttt{DATEINSTALLATION} dans la table \texttt{EMPLOYE}. Vérifier la suppression}
\subsubsection{Requête}
\begin{sql}
ALTER TABLE EMPLOYE DROP COLUMN DATEINSTALLATION;
\end{sql}
\subsubsection{Résultat}
\begin{sql}
Table altered.
\end{sql}
\subsubsection{Vérification}
\begin{sql}
SELECT COLUMN_NAME FROM USER_TAB_COLUMNS WHERE TABLE_NAME = 'CLIENT';

COLUMN_NAME                                                                     
------------------------------                                                  
NUMCLIENT                                                                       
CIV                                                                             
PRENOMCLIENT                                                                    
NOMCLIENT                                                                       
DATENAISSANCE                                                                   
ADRESSE                                                                         
TELPROF                                                                         
TELPRIV                                                                         
FAX                                                                             

9 rows selected.
\end{sql}
\subsection{Renommer la colonne \texttt{ADRESSE} dans la table \texttt{CLIENT} par \texttt{ADRESSECLIENT}. Vérifier}
\subsubsection{Requête}
\begin{sql}
ALTER TABLE CLIENT RENAME COLUMN ADRESSE TO ADRESSECLIENT;
\end{sql}
\subsubsection{Résultat}
\begin{sql}
Table altered.
\end{sql}
\subsubsection{Vérification}
\begin{sql}
SELECT COLUMN_NAME FROM USER_TAB_COLUMNS WHERE TABLE_NAME = 'CLIENT';

COLUMN_NAME                                                                     
------------------------------                                                  
NUMCLIENT                                                                       
CIV                                                                             
PRENOMCLIENT                                                                    
NOMCLIENT                                                                       
DATENAISSANCE                                                                   
ADRESSECLIENT                                                                   
TELPROF                                                                         
TELPRIV                                                                         
FAX                                                                             

9 rows selected.
\end{sql}
\subsection{Ajouter la contrainte suivante : Date de début d’intervention doit être inferieur à la date de fin d’intervention}
\subsubsection{Requête}
\begin{sql}
ALTER TABLE INTERVENTIONS ADD CONSTRAINT CHK_DATEINTERV CHECK (DATEDEBINTERV < DATEFININTERV);
\end{sql}
\subsubsection{Résultat}
\begin{sql}
Table altered.
\end{sql}

\section{Langage de manipulation de données}
\subsection{Remplir toutes les tables par les instances représentées ci-dessus, quels sont les problèmes rencontrés.}
\subsubsection{Requête}
\begin{sql}
INSERT INTO TABLENAME (SCHEMA) VALUES (VALUES);
\end{sql}
\subsubsection{Résultat}
\begin{itemize}
    \item Insertion sans problèmes
    \begin{sql}
    1 row created.
    \end{sql}
    \item le client n 23 ne peut pas etre nee li mois : 13 car inexistant
    \begin{sql}
    INSERT INTO CLIENT VALUES(23,'M','lamine','merabat',to_date('09/13/1965', 'DD/MM/YYYY'),'cite jeanne d arc ecran b2-gambetta-oran','0561724538','0561724538','')
    ERROR at line 1:
    ORA-01843: not a valid month
    \end{sql}
    \item Seule les valeurs \texttt{'MECANECIEN'} et \texttt{'ASSISTANT'} sont autorisés pour la collone \texttt{EMPLOYE.CATEGORIE}
    \begin{sql}
    INSERT INTO EMPLOYE VALUES(80,'lardjoune','karim','',25000)
    ERROR at line 1:
    ORA-01400: cannot insert NULL into ("DBAINTERVENTION"."EMPLOYE"."CATEGORIE") 
    \end{sql}
    \item Les vehicules 77 et 9 n'existe pas d'ou la violation d'integritee \texttt{FK\_INTERVENTIONS\_VEHICULE}
    \begin{sql}
    INSERT INTO VEHICULE VALUES (19,18,77,3904318515,'1985')
    ERROR at line 1:
    ORA-02291: integrity constraint (DBAINTERVENTION.FK_VEHICULE_MODELE) violated - parent key not found 
    
    INSERT INTO VEHICULE VALUES (24,80,9,1789519816,'1998')
    ERROR at line 1:
    ORA-02291: integrity constraint (DBAINTERVENTION.FK_VEHICULE_CLIENT) violated - parent key not found 
    \end{sql}
    \item Les employe 88 et 77 n'existe pas d'ou la violation d'integritee \texttt{FK\_INTERVENANT\_EMPLOYE}
    \begin{sql}
    INSERT INTO INTERVENTIONS VALUES (16,77,'reparation',to_date('2006-06-27 09:00:00','YYYY-MM-DD HH24:MI:SS'),to_date('2006-06-30 12:00:00','YYYY-MM-DD HH24:MI:SS'),25000))
    ERROR at line 1:
    ORA-02291: integrity constraint (DBAINTERVENTION.FK_INTERVENTIONS_VEHICULE) violated - parent key not found 
    
    INSERT INTO INTERVENANT VALUES (14,88,to_date('2006-05-07 14:00:00','YYYY-MM-DD HH24:MI:SS' ),to_date('2006-05-10 18:00:00','YYYY-MM-DD HH24:MI:SS' ))
    ERROR at line 1:
    ORA-02291: integrity constraint (DBAINTERVENTION.FK_INTERVENANT_EMPLOYE) violated - parent key not found 
    \end{sql}
\end{itemize}

\subsection{Supposons que le salaire de l’employé BADI Hatem est augmenté par 5000DA Que faut-il faire ?}
\subsubsection{Requête}
\begin{sql}
UPDATE EMPLOYE SET SALAIRE = SALAIRE + 5000 WHERE PRENOMEMP = 'badi' AND NOMEMP = 'hatem'    
\end{sql}
\subsubsection{Résultat}
\begin{sql}
1 rows updated.
\end{sql}
\subsection{Pour les interventions de mois de Février, ajouter 5 cinq jours à la date de début. Désactiver la contrainte pour autoriser la modification. Réactiver la contrainte.}
\begin{enumerate}
    \item Désactivation de la contraintes
    \subsubsection{Requête}
    \begin{sql}
    ALTER TABLE INTERVENTIONS DISABLE CONSTRAINT CHK_DATEINTERV;
    \end{sql}
    \subsubsection{Résultat}
    \begin{sql}
    Table altered.
    \end{sql}
    \item Modification de la table
    \subsubsection{Requête}
    \begin{sql}
    UPDATE INTERVENTIONS SET DATEDEBINTERV =  DATEDEBINTERV + 5 WHERE EXTRACT(MONTH FROM DATEDEBINTERV) = 2 OR EXTRACT(MONTH FROM DATEFININTERV) = 2 ;
    \end{sql}
    \subsubsection{Résultat}
    \begin{sql}
    4 rows updated.
    \end{sql}
    \item Réactivation de la contrainte
    \subsubsection{Requête}
    \begin{sql}
    CREATE TABLE TABLEERREURS (ADRESSE ROWID, UTILISATEUR VARCHAR2(30), NOMTABLE VARCHAR2(30), NOMCONTRAINTE VARCHAR2(30));
    ALTER TABLE INTERVENTIONS ENABLE CONSTRAINT CHK_DATEINTERV EXCEPTIONS INTO TABLEERREURS
    \end{sql}
    \subsubsection{Résultat}
    \begin{sql}
    ERROR at line 1:
    ORA-02293: cannot validate (DBAINTERVENTION.CHK_DATEINTERV) - check constraint violated     
\end{sql}
On peut pas réactiver la contrainte à cause de violation par queluqes lignes
\begin{sql}
SELECT DATEDEBINTERV, DATEFININTERV FROM INTERVENTIONS WHERE DATEDEBINTERV >= DATEFININTERV;
DATEDEBINTERV      DATEFININTERV
------------------ ------------------
07-MAR-06          26-FEB-06
05-MAR-06          24-FEB-06
04-MAR-06          25-FEB-06
04-MAR-06          22-FEB-06
\end{sql}
\end{enumerate}

\subsection{Supprimer toutes les véhicules de modèle Série 5. Quels sont les problèmes rencontrés.}
\subsubsection{Requête}
\begin{sql}
DELETE VEHICULE WHERE NUMMODELE IN (SELECT NUMMODELE FROM MODELE WHERE MODELE = 'serie 5')
\end{sql}
\subsubsection{Résultat}
\begin{sql}
ERROR at line 1:
ORA-02292: integrity constraint (DBAINTERVENTION.FK_INTERVENTIONS_VEHICULE) violated -
child record found 
\end{sql}
\section{Langage d’interrogation de données}
\subsection{Lister les modèles et leur marque.}
\subsubsection{Requête}
\begin{sql}
SELECT MARQUE , MODELE FROM MARQUE MA, MODELE MO WHERE MA.NUMMARQUE = MO.NUMMARQUE;\end{sql}
\subsubsection{Résultat}
\begin{sql}
    MARQUE                                   MODELE                                 
    ---------------------------------------- ---------------------------------------
    lamborghini                              diablo                                 
    audi                                     série 5                                
    alfa-romeo                               nsx                                    
    mercedes                                 classe c                               
    renault                                  safrane                                
    venturi                                  400 gt                                 
    lotus                                    esprit                                 
    peugeot                                  605                                    
    toyota                                   prévia                                 
    ferrari                                  550 maranello                          
    rolls-royce                              bentley-continental                    
    alfa-romeo                               spider                                 
    maserati                                 evoluzione                             
    porsche                                  carrera                                
    porsche                                  boxter                                 
    volvo                                    s 80                                   
    chrysler                                 300 m                                  
    bmw                                      m 3                                    
    jaguar                                   xj 8                                   
    peugeot                                  406 coupé                              
    venturi                                  300 atlantic                           
    mercedes                                 classe e                               
    lexus                                    gs 300                                 
    cadilac                                  séville                                
    saab                                     95 cabriolet                           
    audi                                     tt coupé                               
    ferrari                                  f 355                                  
    
    27 rows selected.
\end{sql}
\subsection{Lister les véhicules sur lesquels, il y a au moins une intervention.}
\subsubsection{Requête}
\begin{sql}
SELECT V.* FROM VEHICULE V, INTERVENTIONS I WHERE I.NUMVEHICULE = V.NUMVEHICULE;
\end{sql}
\subsubsection{Résultat}
\begin{sql}
    NUMVEHICULE  NUMCLIENT  NUMMODELE   NUMIMMAT ANNE                               
    ----------- ---------- ---------- ---------- ----                               
              1          2          6   12519216 1992                               
              1          2          6   12519216 1992                               
              2          9         20  124219316 1993                               
              3         17          8 1452318716 1987                               
              6         20          6 3853319735 1997                               
              8         16         14 8365318601 1986                               
             10         20         22 9413119935 1999                               
             14         22         21 7543119207 1992                               
             17         12         11 4563117607 1976                               
             20         22          2 1234319707 1997                               
             20         22          2 1234319707 1997                               
             21          3         19 8429318516 1985                               
             22          8         19 1245619816 1998                               
             25         13          5 1278919833 1998                               
             28         10          3 1986219904 1999                               
    
    15 rows selected.
\end{sql}
\subsection{Quelle est la durée moyenne d’une intervention?}
\subsubsection{Requête}
\begin{sql}
SELECT AVG(DATEFININTERV - DATEDEBINTERV) AS DUREE_MOYENNE FROM INTERVENTIONS;\end{sql}
\subsubsection{Résultat}
\begin{sql}
    DUREE_MOYENNE                                                                   
    -------------                                                                   
       -.87222222                                                                       
\end{sql}
\subsection{Donner le montant global des interventions dont le coût d’intervention est supérieur à 30000 DA?}
\subsubsection{Requête}
\begin{sql}
SELECT NUMVEHICULE, SUM(COUTINTERV) FROM INTERVENTIONS GROUP BY NUMVEHICULE HAVING SUM(COUTINTERV) > 30000;
\end{sql}
\subsubsection{Résultat}
\begin{sql}
    NUMVEHICULE SUM(COUTINTERV)                                                     
    ----------- ---------------                                                     
             25           42000                                                     
              1           47000                                                     
              6           40000                                                     
             28           36000                                                     
              2           45000                                                     
             20           54000 
\end{sql}


\end{document}