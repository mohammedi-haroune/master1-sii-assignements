%arara : pdflatex
\documentclass[12pt]{article}

\usepackage{style}

\begin{document}
\def\reportnumber{1}
\def\reporttitle{Introduction au \texttt{ANTLR}}
%----------------------------------------------------------------------------------------
%	TITLE PAGE
%----------------------------------------------------------------------------------------

\begin{titlepage} % Suppresses displaying the page number on the title page and the subsequent page counts as page 1
	\newcommand{\HRule}{\rule{\linewidth}{0.5mm}} % Defines a new command for horizontal lines, change thickness here
	
	\center % Centre everything on the page
	
	%------------------------------------------------
	%	Headings
	%------------------------------------------------
	
	\baselineskip=2\baselineskip 
	\textsc{\LARGE Université des Sciences et de la Technologie Houari Boumediene}%\\[1cm] % Main heading such as the name of your university/college

	%------------------------------------------------
	%	Logo
	%------------------------------------------------
	
	%\vfill\vfill
	\vfill
	\includegraphics[width=0.3\textwidth]{USTHB_Logo.png}\\[1cm] % Include a department/university logo - this will require the graphicx package
	 
	%----------------------------------------------------------------------------------------
	
	\textsc{\Large Compilation}\\[0.5cm] % Major heading such as course name
	%\textsc{\large Minor Heading}\\[0.5cm] % Minor heading such as course title
	
	%------------------------------------------------
	%	Title
	%------------------------------------------------
	
	\HRule\\[0.4cm]
	\baselineskip=1.2\baselineskip 
	{\huge\bfseries Travaux Pratique  \textdegree  \reportnumber \\ \reporttitle}\\[0.4cm] % Title of your document
	
	\HRule\\[1.5cm]
	
	%------------------------------------------------
	%	Author(s)
	%------------------------------------------------
	
		\begin{center}
			\large
			\textit{Professeur}\\
			% votre non ICI
		\end{center}
	
	%------------------------------------------------
	%	Date
	%------------------------------------------------
	
	\vfill\vfill\vfill % Position the date 3/4 down the remaining page
	
	{\large\today} % Date, change the \today to a set date if you want to be precise
	
	
	\vfill % Push the date up 1/4 of the remaining page
	
\end{titlepage}

\section{Installation}
Dans cette section vous trouvez toutes les étapes a suivre pour l'installation de l'outil \texttt{ANTLR} sous les différent système d'exploitation.

La procédure d'installation est vraiment simple et peut se resumer dans ces trois étpaes : 
\begin{enumerate}
    \item La première étape c'est de télécharger le fichier \texttt{jar} de l'outil.
    \item La prochaine étpae c'est de mettre le fichier dans le \texttt{CLASSPATH} pour que java peut le trouver. 
    \item La dernière étape consiste à créer un alias pour le programme de tel façon ou on peut l'appeler facilement.
\end{enumerate}



\subsection{Installation sous Linux}
Exécuter la suite de commande suivantes dans un terminal \shellcmd{bash}

\begin{enumerate}
    \item 
    \shellcmd{cd /usr/local/lib}

    \shellcmd{sudo curl -O http://www.antlr.org/download/antlr-4.7-complete.jar}
    \item \shellcmd{export CLASSPATH=".:/usr/local/lib/antlr-4.0-complete.jar:\char36CLASSPATH"}
    \item \shellcmd{alias antlr4="java -jar /usr/local/lib/antlr-4.0-complete.jar"}

\end{enumerate}

Si vous voulez que ces configuations soit glabale et qu'il reste après la fermuture du \shellcmd{bash} il faut ajouter la commande \shellcmd{export} dans le fichier \shellcmd{.bash\_profile} et la commande \shellcmd{alias} dans le fichier \shellcmd{.bashrc}, les deux derniers fichiers se trouve dans le dossier \shellcmd{HOME}. 

Et enfin executer les commandes : \shellcmd{bash .bash\_profile} et \shellcmd{bash .bashrc} pour executer les fichires précedentes.

\end{document}
